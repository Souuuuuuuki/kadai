\documentclass[dvipdfmx,10pt]{jlreq}
\usepackage[top=30truemm,bottom=30truemm,left=15truemm,right=15truemm]{geometry}
\usepackage{url}

\begin{document}

\title{第12回予習復習問題}
\author{2025311066 藤井壮樹}
\date{\today}
\maketitle

\section{第10回予習復習問題の確認・修正}
\subsection*{1.}
確認しました.
\subsection*{2.}
確認しました.
\subsection*{3.}
確認しました.

\section{第12回予習復習問題}
\subsection*{4.}
\begin{itemize}
  \item まず,2分探索木と同様に削除を行う.
  \item バランスを崩れたかどうかをチェックする.
  \begin{itemize}
    \item 崩れていなければ終了.
    \item 崩れていれば,AVL木になるよう再構成する.
  \end{itemize}
\end{itemize}
\subsection*{5.}
\begin{itemize}
  \item 定義が違う.(AVL木はバランスを保っている.)
  \item 計算量が違う.
  \item データの操作の仕方が違う.
\end{itemize}
\subsection*{6.}
バランスが取れた多分木であるが,AVL木と違って各ノードが複数のカギを持つ.木の高さを極端に低くできるという特徴がある.

\begin{thebibliography}{9}
	\bibitem{文献1} 第12回授業資料
  \bibitem{文献2} 平衡木の完全ガイド \today 閲覧 \url{https://everplay.jp/column/20882}
\end{thebibliography}

\end{document}