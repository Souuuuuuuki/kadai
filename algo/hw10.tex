\documentclass[dvipdfmx,10pt]{jlreq}
\usepackage[top=30truemm,bottom=30truemm,left=15truemm,right=15truemm]{geometry}
\usepackage{url}

\begin{document}

\title{第10回アルゴリズムとデータ構造 課題}
\author{2025311066 藤井壮樹}
\date{\today}
\maketitle

\section{第8回予習復習問題の確認・修正}
\subsection*{1.}
リングバッファを使うことで実装することができる.
\subsection*{2.}
確認しました.
\subsection*{3.}
確認しました.
\subsection*{4.}
m分木において,幅優先探索の順に配列を入れることで実現することができる(補足).

\section{第10回予習復習問題}
\subsection*{5.}
まずは削除要素を探索する.探索失敗であれば失敗を返して終了する.\\
探索に成功した場合,以下の手順で削除を行う.
\begin{itemize}
 \item その要素を削除する.
 \item 木の再構成を行う.
\end{itemize}
なお,木の再構成をする際は,「再構成した木が二分木であること」,「計算量ができるtだけ少ないこと」の条件を満たす必要がある.\\
\subsection*{6.}
AVL木とは,通常の二分木を拡張して,最悪計算量をO(log n)に抑えることができるようにしたデータ構造である.\\

\subsection*{7.}
挿入と削除の後で各ノードでのバランス(左の木の高さから右の木の高さを引いた値)を回復する操作を行う必要がある.\\

\begin{thebibliography}{}
	\bibitem{文献1} アルゴリズムとデータ構造, 第9回講義資料
  \bibitem{文献2} AVL木 \url{https://www-ui.is.s.u-tokyo.ac.jp/~takeo/course/2016/algorithm/AVL.pdf}
\end{thebibliography}

\end{document}