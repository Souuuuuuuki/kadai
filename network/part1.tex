\documentclass[dvipdfmx,10pt]{jlreq}
\usepackage[top=30truemm,bottom=30truemm,left=15truemm,right=15truemm]{geometry}


\begin{document}

\title{第一回 ネットワーク勉強会 課題}
\author{2025311066 藤井壮樹}
\date{\today}
\maketitle

\section{第一部 ネットワークの基礎知識}
\subsection{キーワード解説}
\subsubsection{プロトコル}
プロトコルとは,通信において情報の送受信を行うためのルールや手順のことである.
ここにはデータの形式やエラーチェックの方法,接続の確立と終了の手順などの要素が含まれている.
\subsubsection{インターネットワーキング}
ネットワーク同士を接続しあうこと.
単一の巨大なネットワークではなく,複数を結合して作ることで不要な通信を全体に広めることなくとどめられる.
\subsubsection{コネクションレス型通信}
相手との接続を確立させずにデータを送信すること.
通信の信頼性は保証されないが,通信にかかる負荷が低く遅延が少ないという特徴がある.

\subsubsection{ユニキャスト/ブロードキャスト}
ユニキャストとは,単一のアドレスを指定することで1対1で行われるデータ通信のことである.Webサイトの閲覧やメールの送受信などが用途として挙げられる.\\
ブロードキャストとは,1対不特定多数で行われるデータ通信である.ネットワーク上にファイルを転送する場合などに用いられる.

\subsubsection{物理アドレス/MACアドレス}
物理アドレスとは,1つの端末に1つしか存在しない,機器自体に紐づけられたものである.

\subsection{記述}
\subsubsection{SSLの通信はコネクション指向型orコネクションレス型か}
コネクション指向型?visionやロボットと戦略pcの通信は1対1のように感じるから.

\subsubsection{SSLの通信からブロードキャスト通信の例}
GCと各ロボットの間の通信

\subsection{ハンズオン}
ipconfig /allコマンドの実行結果から,以下の情報を読み取った.

\section{第2部:IPアドレスと役割}
\subsection{キーワード解説}
\subsubsection{IPアドレス}
個別のネットワークや機器を識別するための番号,住所.

\subsubsection{ネットワーク部とホスト部}
IPアドレスはネットワーク部とホスト部に分かれている.ネットワーク部はそのIPアドレスが属しているネットワークを識別する部分であり,ホスト部はコンピュータを識別するための部分.

\subsubsection{サブネットマスク}
ネットワークを複数の物理ネットワークに分割するのに使用する,番号のこと.

\subsubsection{プライベートIPアドレス}
過程や企業などの組織内のみで割り当てられるIPアドレスのこと.

\subsubsection{CIDR(サイダー)表記}
198.51.100.xxx/24のような形で,「/」を使用して「IPアドレス/サブネットマスク」を表記する方法.表記の簡略化が目的である.

\subsubsection{ネットワークアドレス}
ネットワークに割り当てられたIPアドレスの中で,ネットワーク地震を指し示すアドレス.

\subsubsection{ブロードキャストアドレス}
ブロードキャスト通信をする際に使われる,パケットの宛先アドレス.

\subsection{記述}



%\begin{thebibliography}{}
%	\bibitem{文献1}
%\end{thebibliography}

\end{document}